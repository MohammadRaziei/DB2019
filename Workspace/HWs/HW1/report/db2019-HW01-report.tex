% !TEX program = xelatex
\documentclass[12pt]{article}

\usepackage{hyperref}

\usepackage{amsmath,amsfonts,amssymb}
\usepackage{amsthm}
%\usepackage{bidipoem} % for poem -> traditionalpoem
%\usepackage{longtable} % for longtables

%\usepackage{pgf,tikz,pgfplots}
%\pgfplotsset{compat=1.15}
%\usepackage{mathrsfs}
%\usetikzlibrary{arrows}
%%\pagestyle{empty}


\usepackage{xepersian}
\settextfont{Yas}
\setdigitfont{Yas}

\defpersianfont\nast{IranNastaliq}
\defpersianfont\sols{XB Sols}
\defpersianfont\naz{B Nazanin}
\defpersianfont\yas{Yas}


\title{پاسخ تمرین اول پایگاه داده}
\author{محمد رضیئی فیجانی\\ 9423052}

%\newtheorem{thm}{قضیه}[section]
%\newtheorem{Def}{تعریف}[section]
%\newtheorem{test}{تمرین}
\newtheorem{question}{سوال}
%\newcommand{\dd}{\, \mathbf{d} }


\begin{document}
\maketitle

\section{پاسخ سوالات}\label{chpt1}
\begin{question}
	چون هر زیر مجموعه غیر تهی از k کلید کاندیدا میتواند یک ابرکلید باشد، بنابراین تعداد 
	$2^k - 1$
	ابرکلید خواهیم داشت.
\end{question}
\begin{question}
درستی و نادرستی در زیر بررسی شده است.
\begin{enumerate}
	\item
	درست است.
	
	اگر روی دو مجموعه به صورت جداگانه عمل select را انجام دهیم و سپس روی نتیجه حاصل اجتماع بگیریم پاسخ با حالتی که روی اجتماع دو جدول select کنیم، تفاوتی ندارد.
	
	\item
	نادرست است.
	
	اگر ابتدای projection انجام دهیم؛ ممکن است ستونی را که میخواهیم روی آن select کنیم، انتخاب نشده باشد. از این رو نتیجه با سمت دیگر تساوی، برابر نخواهد بود. معمولا از سمت راست تساوی استفاده می شود.
	
	\item
	درست است.
	
	دو مجموعه R  و S با یک‌دیگر اشتراکی دارند که در این قسمت، بعضی از اعضای آن شرط $\theta$ را برآورده می کنند.
	
	\item
	نادرست است.
	
	زیرا اشتراک هر دو مجموعه ای، زیرمجموعه هریک از آن‌هاست. بنابراین کلید کاندیدای M یا N می‌تواند به تنهایی به‌عنوان کلید کاندیدای $M \cap N$ باشد.
	
	\item

	
	\item
	
	
	
	
	
\end{enumerate}
\end{question}
\begin{question}
	
\end{question}
\begin{question}
	
\end{question}
\begin{question}
	
\end{question}

\section{github}\label{chpt2}
تمامی فایل های تمرینات این درس در آدرس زیر در 
\href{https://github.com/mohammadraziei/db2019/}{github}
قابل دسترسی می باشد.

\begin{latin}
\begin{center}
{\href{https://github.com/mohammadraziei/db2019/}{https://github.com/mohammadraziei/db2019/}}
\end{center}
\end{latin}


\end{document}